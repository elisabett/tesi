\chapter{Realizzazione del progetto}
\thispagestyle{empty}

Gli strumenti coinvolti nel progetto sono quelli descritti nei capitoli precedenti, il monocromatore Jobin-Yvon HR460 e la fotocamera intensificata 4 Picos. Di seguito si descrive il programma realizzato per il controllo dei due strumenti attraverso un unico programma.

\section{Approccio iniziale}
Avendo davanti un progetto gi� scritto e funzionante, anche se per una fotocamera diversa, il primo approccio � stato quello di cercare di integrare il vecchio programma sostituendo i subVI della fotocamera precedente con quelli della fotocamera attuale. Tale tentativo si � rivelato fin dall'inizio fallimentare in quanto il codice del vecchio progetto risulta troppo caotico e avviluppato per essere modificato senza operare ingenti cambiamenti. Si � quindi optato per una riscrittura completa del codice prendendo sempre spunto da quello gi� scritto per le parti riutilizzabili.

\section{Struttura del progetto}
\textbf{Inserire foto del progetto .lvproj}
Per rendere il programma maggiormente organico e comprensibile si � deciso di creare un progetto LabVIEW: in questo modo i file e i subVI coinvolti sono raggruppati in un unico progetto, che risulta sicuramente pi� gestibile di un semplice insieme di file.\\
Si � creato all'inizio un VI principale, un \textit{main} che ha lo stesso scopo dei \textit{main} degli altri linguaggi di programmazione. In seguito si � cercato di creare pi� subVI possibile per rendere il programma leggibile e ordinato. Si descriveranno pi� avanti i subVI contenuti nel progetto e i loro scopi.

\section{Front Panel}
\textbf{Inserire foto del progetto .lvproj}
La struttura pensata per il front panel di questo progetto � quella dell'interfaccia grafica a \textit{tab} (controllo grafico di navigazione che permette all'utente di muoversi da un gruppo di controlli a un altro). Tale widget permette un'ideale suddivisione delle diverse aree di competenza del programma: di seguito verranno elencate con una descrizione degli elementi contenuti da ognuna.
\begin{itemize}
	\item \textit{Setting Mono} contiene i controlli necessari al settaggio del monocromatore. \textit{Grating} per impostare il grating (1200 o 2400) e rispettivi controlli per specificarne un eventuale calibrazione. \textit{Set Entrance Slit} e \textit{Set Wave Length} per impostare la slit di entrata e la lunghezza d'onda. \textit{Actual WL} e \textit{Slit Width} per visualizzare lunghezza d'onda e slit effettivamente inserite. Sono presenti inoltre tre led: \textit{Settings Error} indica un errore generico di settaggio, \textit{Invalid Wavelength} indica l'inserimento di una lunghezza d'onda non valida e \textit{Limits hit} evidenzia l'inserimento di un valore oltre i limiti consentiti.
	\item \textit{Acquire Data} contiene tutti i controlli per poter impostare la fotocamera e acquisire i dati. 
	\item \textit{Post Processing} permette di visualizzare lo spettro corrispondente all'immagine acquisita ed effettuare dei calcoli su tali dati. Con \textit{Waveform Integral} � possibile calcolare l'integrale della lunghezza d'onda su un certo intervallo. \textit{Peak Detector} consente di visualizzare il numero di picchi trovati al di sopra di una certa soglia di inserire e i loro valori. Attraverso il controllo \textit{Pixel To Average} � possibile scegliere il numero di pixel da con cui fare la media da visualizzare poi nel grafico. Sono stati mantenuti anche i controlli per acquisire il background e sottrarlo alla foto.
	\item \textit{Save} presenta due controlli booleani: \textit{Save} salva i dati be formattati in un file di testo che pu� essere aperto successivamente con altri editor, \textit{Save Image} salva invece l'immagine vera e propria.
\end{itemize}
Al di fuori di questo controllo � presente un pannello dedicato all'inizializzazione degli strumenti. Il led \textit{Init} segnala che il monocromatore � stato inizializzato senza errori e viene visulizzata anche la versione del programma, in caso contrario si attiva il led \textit{Error Occured}. Sono presenti anche due controlli per specificare la connessione che si intende utilizzare verso la fotocamera: \textit{USB} per connetterla al calcolatori attraverso cavo USB, \textit{CameraLink} o \textit{Analog} per altre modalit� di connessione non contemplate in questo progetto.\\
Infine il bottone \textit{Exit} consente di arrestare il programma nel modo corretto.

\section{Block Diagram}