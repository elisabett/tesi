\chapter{Fase preliminare alla realizzazione del progetto}
\thispagestyle{empty}

Per entrare in confidenza con il linguaggio di programmazione grafico LabVIEW si � reso necessario partire da un esempio fornito con le API LabVIEW legate alla nuova fotocamera intensificata.

\section{Breve descrizione dello strumento}

\section{Descrizione dell'esempio fornito}

L'esempio fornito � molto semplice e non contiene tutte le opzioni disponibili per la fotocamera. Analizziamo come fatto in precedenza i due componenti principali del programma: front panel e block diagram.

\subsection{Front Panel}
\textbf{Immagine front panel esempio}
Nel front panel possiamo vedere due controlli che � obbligatorio definire prima della partenza del VI, e sono il tipo di connessione \textit{Connection Type} e la porta seriale \textit{Serial Port} nel caso in cui il tipo di connessione sia \textit{Analog}. � infatti permesso scegliere fra tre tipi di connessione allo strumento: \textit{USB} (che � la connessione usata nel nostro caso), \textit{CameraLink} o \textit{Analog}. A fianco di questi due controlli troviamo due bottoni booleani per acquisire immagini in un singolo frame oppure in modo continuo. � poi possibile inserire il tempo di esposizione \textit{Exposure Time}, il numero di frame da catturare \textit{Frames to accumulate}, e modificare il valore \textit{Max. displayed value} del fondo scala. Un indicatore di immagine rende possibile la visualizzazione dell'immagine e un bottone booleano permette di salvare l'immagine. Infine troviamo un controllo per terminare il programma.

\subsection{Block Diagram}
\textbf{Immagine block diagram esempio}
Si pu� considerare il bock diagram idealmente diviso in tre parti che eseguono in senquenza: una parte iniziale di inizializzazione, una intermedia di funzionamento del programma, e una finale di chiusura graceful dell'applicazione.\\
Il primo subVI invocato � \emph{ConnectCamera.vi} e deve essere invocato prima che la fotocamera sia utilizzata, imposta la connessione della fotocamera dipendentemente dall'input immesso dall'utente; restituisce in uscita un cluster contenente tutte le informazioni che riguardano le impostazioni della fotocamera. Tale cluster di dati, insieme al cluster di errore, entra in una flat sequence che recupera le impostazioni memorizzate da un uso antecedente dello strumento e le mostra nel controlli appositi nel front panel; viene anche impostato il bottone booleano \textit{Live Mode} a false.\\
In seguito alla flat sequence troviamo un ciclo while con lo scopo di individuare il cambiamento dei bottoni e dei controlli. All'interno di questo ciclo sono presenti due case structure: la prima associata al bottone \textit{Save Image...} che permette di salvare l'immagine, la seconda associata all'\textit{or} fra il valore dei due bottoni \textit{Single Exposure} e \textit{Live Mode}. Viene eseguito quindi il codice all'interno di quest'ultima case structure qualora o un bottone o l'altro esclusivamente (grazie ad un'altra case structure interna) abbiamo il valore true. Una volta dentro la case structure viene eseguito un ciclo while all'interno del quale vengono invocate alcune API LabVIEW della fotocamera: dapprima \emph{SetExposureTime.vi} con la quale si setta il tempo di esposizione, poi \emph{GetExposure.vi} che restituisce la matrice di pixel corrispondente alla foto. Attraverso alcuni calcoli che tengono in considerazione anche il valore di fondo scala inserito viene generata l'immagine da visualizzare del controllo destinato. Il ciclo while interno si arresta se viene presa una singola immagine, oppure se non si � selezionata la modalit� Live Mode, o se si � premuto il bottone \textit{Exit} per uscire dal ciclo o quello di salvataggio dell'immagino, o ancora se si � verificato qualche errore. Il ciclo esterno si arresta invece se si � verificato qualche errore o se si � invocato il bottone \textit{Exit} per fermarlo.\\
All'esterno di tale ciclo viene eseguito l'ultimo frammento di codice: in una flat sequence vengono resettati i bottoni di \textit{Exit} e \textit{Live Mode}, impostandoli a false, e infine viene invocato l'ultimo subVI relativo alla fotocamera, \emph{ExitCamera.vi}, che ha il compito di rilasciare le risorse impegnate dal programma e arrestare la fotocamera nel modo corretto.