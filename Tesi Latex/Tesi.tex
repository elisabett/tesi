\documentclass[12pt,a4paper,openright,twoside]{report}

\usepackage[italian]{babel}

\usepackage[latin1]{inputenc}
% libreria per impostare il documento
\usepackage{fancyhdr}
% libreria per avere l'indentazione
% all'inizio dei capitoli, ...
\usepackage{indentfirst}
% libreria per mostrare le etichette
%\usepackage{showkeys}
% libreria per inserire grafici
\usepackage{graphicx}
% libreria per utilizzare font
%   particolari ad esempio
%   \textsc{}

\usepackage{listings}

\usepackage{newlfont}
% librerie matematiche
\usepackage{amssymb}
\usepackage{amsmath}
\usepackage{latexsym}
\usepackage{amsthm}

\oddsidemargin=30pt \evensidemargin=20pt%impostano i margini
\hyphenation{sil-la-ba-zio-ne pa-ren-te-si}%serve per la sillabazione: tra parentesi 
					   %vanno inserite come nell'esempio le parole 
%					   %che latex non riesce a tagliare nel modo giusto andando a capo.

% comandi per l'impostazione
%   della pagina, vedi il manuale
%   della libreria fancyhdr
%   per ulteriori delucidazioni
\pagestyle{fancy}\addtolength{\headwidth}{20pt}
\renewcommand{\chaptermark}[1]{\markboth{\thechapter.\ #1}{}}
\renewcommand{\sectionmark}[1]{\markright{\thesection \ #1}{}}
\rhead[\fancyplain{}{\bfseries\leftmark}]{\fancyplain{}{\bfseries\thepage}}
\cfoot{}
\linespread{1.3} %comando per impostare l'interlinea
% definisce nuovi comandi

\begin{document}

\include{frontespizio}
\clearpage{\pagestyle{empty}\cleardoublepage}
\begin{titlepage}                       %crea un ambiente libero da vincoli
                                        %   di margini e grandezza caratteri:
                                        %   si pu\`o modificare quello che si
                                        %   vuole, tanto fuori da questo
                                        %   ambiente tutto viene ristabilito
\thispagestyle{empty}                   %elimina il numero della pagina
\topmargin=6.5cm                        %imposta il margina superiore a 6.5cm
\raggedleft                             %incolonna la scrittura a destra
\large                                  %aumenta la grandezza del carattere
                                        %   a 14pt
\em
Con ogni umiltà e mansuetudine, con pazienza, sopportandovi gli uni gli altri nell’amore.\\
\textit{Ef 4,2}                                    %emfatizza (corsivo) il carattere
Alla mia famiglia.                      
\newpage                                %va in una pagina nuova
\clearpage{\pagestyle{empty}\cleardoublepage}
\end{titlepage}
\clearpage{\pagestyle{empty}\cleardoublepage}
\pagenumbering{roman}                   %serve per mettere i numeri romani


\chapter*{Introduzione}                 %crea l'introduzione (un capitolo
                                        %   non numerato)
%%%%%%%%%%%%%%%%%%%%%%%%%%%%%%%%%%%%%%%%%imposta l'intestazione di pagina
\rhead[\fancyplain{}{\bfseries
INTRODUZIONE}]{\fancyplain{}{\bfseries\thepage}}
\lhead[\fancyplain{}{\bfseries\thepage}]{\fancyplain{}{\bfseries
INTRODUZIONE}}
%%%%%%%%%%%%%%%%%%%%%%%%%%%%%%%%%%%%%%%%%aggiunge la voce Introduzione
                                        %   nell'indice
\addcontentsline{toc}{chapter}{Introduzione}
Nell'ambito dello studio dei plasmi generati tramite scariche elettriche (?) si rivela necessario catturare gli spettri generati da suddette interazioni per studiarne diverse propriet� quali composizione e temperatura. Per effettuare tali misurazioni sono necessari due strumenti: un monocromatore che scomponga e rifletta la fonte di luce in input e una fotocamera che catturi l'immagine dello spettro. \\
Risulta quindi necessaria un'interfaccia utente in grado di controllare simultaneamente i due strumenti in modo da facilitare il percorso di acquisizione e manipolazione degli spettri. Lo scopo della tesi � implementare l'interfaccia richiesta attraverso il linguaggio di programmazione grafica LabVIEW, con l'ausilio delle API dei due strumenti.\\
� presente un progetto preesistente che realizza l'interfaccia di controllo del monocromatore e di una fotocamera non intensificata. A partire da questo progetto se ne � realizzato uno nuovo che riutilizza alcune parti del precedente, considerando lo stesso monocromatore e una nuova fotocamera intensificata.\\
A partire quindi dal programma esistente si � effettuato uno studio dello stesso per poterlo cos� adattare alla nuova fotocamera. Dopo un primo tentativo di riutilizzo e refactoring si � preferito realizzare un programma ex novo, modificando anche alcuni elementi dell'interfaccia grafica. Chiaramente si sono mantenuti alcuni elementi, in particolare quelli legati al monocromatore, dal momento che lo strumento � rimasto lo stesso cos� come le librerie ad esso associate.
%non numera l'ultima pagina sinistra
\clearpage{\pagestyle{empty}\cleardoublepage}
\tableofcontents                        %crea l'indice
%imposta l'intestazione di pagina
%\rhead[\fancyplain{}{\bfseries\leftmark}]{\fancyplain{}{\bfseries\thepage}}
\lhead[\fancyplain{}{\bfseries\thepage}]{\fancyplain{}{\bfseries
INDICE}}
%non numera l'ultima pagina sinistra
\clearpage{\pagestyle{empty}\cleardoublepage}
\listoffigures                          %crea l'elenco delle figure
%non numera l'ultima pagina sinistra
\clearpage{\pagestyle{empty}\cleardoublepage}
\listoftables                           %crea l'elenco delle tabelle
%non numera l'ultima pagina sinistra
\clearpage{\pagestyle{empty}\cleardoublepage}
\pagenumbering{arabic} 

\chapter{Documento dei requisiti}
\label{capitolo1}
\thispagestyle{empty}

Si descrive nelle seguenti sezioni la specifica dei requisiti del sistema software che si intende sviluppare.

\section{Scopo}
Come accennato nell'introduzione si rende necessaria la realizzazione di una nuova interfaccia che possa controllare simultaneamente il monocromatore HR640 e la fotocamera intensificata 4 Picos. Il precedente programma aveva la capacit� di controllare lo stesso monocromatore ma una diversa fotocamera, SensiCam (version 3.0).

\section{Descrizione generale}
L'utente, attraverso il programma da progettare, deve poter controllare sia il monocromatore che la fotocamera intensificata. � importante che questi due strumenti possano essere controllati simultaneamente poich� le loro funzioni sono strettamente legate: il monocromatore genera un determinato spettro attraverso la sua struttura interna e la fotocamera deve poter catturare tale spettro il pi� precisamente possibile.

\subsection{Funzioni del prodotto}
Attraverso il programma deve essere possibile inizializzare le principali caratteristiche del monocromatore in base alle esigenze dell'utente: grating, lunghezza d'onda e fenditura di entrata. Deve anche essere possibile settare e modificare le impostazioni della fotocamera, compatibilmente con le funzioni messe a disposizione tramite le API della stessa. 

\subsection{Caratteristiche utente}
Il progetto da realizzare viene utilizzato da professori e dottorandi per la caratterizzazione spettroscopica di plasmi a bassa pressione e atmosferici. � necessario trovare un punto di equilibrio tra le esigenze dell'utente e i vincoli di programmazione.

\section{Requisiti specifici}

\subsection{Requisiti dell'interfaccia utente}
L'interfaccia deve risultare il pi� intuitiva possibile, per questo si deve cercare di mantenere un minimo di omogeneit� con il programma passato. Devono infatti coesistere nella stessa finestra sia i controlli per il monocromatore che quelli per la fotocamera.\\Vediamo quali sono i requisiti specifici per i due strumenti.

\begin{description}
  \item[Monocromatore] Per quanto riguarda la parte di interfaccia che permette di impostare correttamente il monocromatore � necessario poter modificare il grating (1200 o 2400) e la calibrazione di correzione della lunghezza d'onda. Si deve anche poter modificare la slit di entrata e la lunghezza d'onda;
  \item[Fotocamera] La parte di interfaccia che concerne la fotocamera intensificata deve contenere i seguenti parametri:
	\begin{itemize}
\item MCP Gain Voltage: guadagno di tensione in volt tra 600 V e 950 V
\item Exposure Time: tempo di esposizione, deve essere positivo e si deve poter specificare l'unit� di misura
\item Delay: ritardo in secondi dell'acquisizione dell'immagine
\item Frames to accumulate: numero di immagini da acquisire
\item Trigger Source: -Trig per far scattare l'otturatore esternamente attraverso l'input di trigger negativo collegato elettricamente ad un segnale, Fsync per far scattare l'otturatore internamente dal segnale Fsync
\item Gate Control: interno, l'input di controllo dell'otturatore � connesso internamente all'output IntGtP che pu� essere usato per scopi di controllo o trigger, o esterno, l'input di controllo dell'otturatore � collegato al connettore in input di ExtGtP
\item Start Option: Cold Start, non ci sono parametri pregressi da utilizzare, Warm Start, dopo il riavvio della fotocamera si chiede all'utente se si devono caricare i parametri usati precedentemente, Auto Warm Start, dopo il riavvio della fotocamera vengono caricati i parametri usati in precedenza senza la conferma da parte dell'utente
\item Detector: per specificare l'area attiva del chip CCD, frame intero, binning 2x2 oppure ROI (Region of Interest, solo 1/3 dell'area complessiva)
\item Digitalizer Mode: acquisizione dell'immagine con una precisione di 8 o 14 bit per pixel
\item CCD Gain: automatico oppure settabile manualmente (in dB)
\item Trigger Mode: trigger diretto (ad ogni segnale di trigger far� scattare l'otturatore) oppure singolo (in questo caso deve essere possibile inserire il numero di trigger per frame)
\item Modalit� di acquisizione dell'immagine: Single Exposure per catturare una sola immagine, Live Mode per acquisire immagini in modo continuo;
\end{itemize}
\end{description}
Devono anche essere presenti dei comandi per il post-processing e l'analisi delle immagini acquisite. Come nel progetto preesistente � necessario visualizzare lo spettro in un grafico che abbia nell'asse orizzontale il range di lunghezza d'onda osservato e nell'asse verticale l'intensit� relativa dello spettro. Inoltre deve essere possibile sottrarre un background acquisito in precedenza ed effettuare dei calcoli sullo spettro, come integrale sulla lunghezza d'onda, FWHM, picco massimo e numero di picchi. Alla fine delle operazioni, qualora ce ne sia la possibilit�, si deve poter salvare i dati in un file di testo che tenga traccia dei valori calcolati per visualizzare lo spettro.

\subsection{Linguaggio di programmazione}
Il linguaggio di programmazione da utilizzare per realizzare il progetto richiesto � LabVIEW.\\ LabVIEW � un linguaggio di programmazione grafica (G - Graphical Programming Language) che utilizza un modello a flusso di dati invece di linee sequenziali di codice testuale, permettendo di scrivere codice funzionale utilizzando un layout grafico che assomiglia a un diagramma di flusso. \\
L'ambiente di programmazione LabVIEW presenta due principali finestre di lavoro: front panel e block diagram. 
\begin{description}
\item[Front Panel]Finestra in cui si visualizza e si modifica l'interfaccia utente. Sono disponibili diversi stili e comandi per l'interazione con l'utente. Si possono inserire bottoni, caselle di inserimento di stringhe o numeri, led, controlli a tab, indicatori per visualizzare grafici, immagini, matrici ecc.
\item[Block Diagram]Finestra in cui si compone il codice a blocchi con controlli, indicatori, subVI e moltre alte componenti. Da tale finestra � possibile effettuare il debug del codice e controllare il contenuto delle variabili.
\end{description}
Le strutture principali utilizzate nel codice sono cicli, while e for, e case structure, che funzionano come if o switch. Per mantenere il valore delle variabili e condividerle con altre strutture si usano fili che le interconnettono. Elementi fondamentali nel linguaggio di programmazione grafica sono controlli e indicatori: i controlli permettono all'utente di inserire o modificare i dati, mentre gli indicatori consentono di visualizzare lo stato delle variabili.
\chapter{Analisi del progetto preesistente}
\thispagestyle{empty}
\lhead[\fancyplain{}{\bfseries\thepage}]{\fancyplain{}{\bfseries\rightmark}}
\pagenumbering{arabic} 

\section{Breve descrizione degli strumenti coinvolti}

\subsection{Monocromatore}
Un monocromatore è un dispositivo che scompone un singolo fascio di luce policromatica in più fasci di luce monocromatica (che contiene cioè onde di una sola frequenza), permettendo così di analizzare l'intensità in funzione della lunghezza d'onda.\\

Nello strumento la luce policromatica entra da una fessura; tramite un sistema ottico viene inviata su un reticolo di diffrazione o ad un prisma che scompone il fascio. Una seconda fenditura raccoglie poi il fascio di una determinata lunghezza d'onda.\\
In questo progetto si considera un monocromatore Jobin-Yvon HR460. La luce che entra dalla fessura viene rifratta (?) due volte attraverso due lenti tra cui è interposta una griglia che scompone il fascio di luce. 

\subsection{Fotocamera}
Una fotocamera intensificata è una fotocamera che al posto della pellicola fotosensibile utilizza un sensore (CCD) in grado di catturare l'immagine e trasformarla in un segnale elettrico di tipo analogico. Gli impulsi elettrici vengono convertiti in digitale da un convertitore analogico/digitale in un chip di elaborazione esterno al sensore. Viene quindi generato un flusso di dati digitali atti ad essere immagazzinati in vari formati su supporto di memoria.\\
Il CCD (Charged-Coupled Device) è un dispositivo attraverso il quale si ottiene un segnale elettrico in uscita, in seguito a una sequenza temporizzata di impulsi, grazie al quale è possibile ricostruire la matrice di pixel che compongono l'immagine proiettata sulla superficie del CCD stesso. Questa informazione può essere usata come segnale analogico, e quindi essere usata per riprodurre l'immagine su un monitor, oppure può essere convertita in formato digitale.
\section{Principali funzioni}


\subsection{Monocromatore}
\subsection{Fotocamera}

\section{Considerazioni sul progetto a livello di programmazione}

In pratica i difetti dai.
\chapter{Fase preliminare alla realizzazione del progetto}
\thispagestyle{empty}

Prima della realizzazione del progetto per controllare entrambi gli strumenti, si � optato per la scrittura di due progetti separati. In questo modo � stato possibile verificare la versione del codice pi� efficacie per ogni strumento, cos� da poter unire in seguito le due logiche, evitando conflitti.

\section{Fotocamera}

Per entrare in confidenza con il linguaggio di programmazione grafico LabVIEW e con lo strumento, si � reso necessario partire da un esempio fornito con le API LabVIEW legate alla nuova fotocamera intensificata.

\subsection{Breve descrizione dello strumento}
La fotocamera intensificata per cui si deve scrivere il programma che la controlli � la 4 Picos con CCD intensificato di Stanford Computer Optics, Inc.

\subsection{Descrizione dell'esempio fornito}

L'esempio fornito � molto semplice e non contiene tutte le opzioni disponibili per la fotocamera. Analizziamo come fatto in precedenza i due componenti principali del programma: front panel e block diagram.

\subsubsection{Front Panel}
\begin{figure} [h]
	\includegraphics[width=\linewidth]{img/SCOexample_fp.png}
	\caption{Front Panel del progetto per sola fotocamera}
	\label{fig:midProj_camera_fp}
\end{figure}
Nel front panel in figura \ref{fig:midProj_camera_fp} possiamo vedere due controlli che � obbligatorio definire prima della partenza del VI, e sono il tipo di connessione \textit{Connection Type} e la porta seriale \textit{Serial Port} nel caso in cui il tipo di connessione sia \textit{Analog}. � infatti permesso scegliere fra tre tipi di connessione allo strumento: \textit{USB} (che � la connessione usata nel nostro caso), \textit{CameraLink} o \textit{Analog}. A fianco di questi due controlli troviamo due bottoni booleani per acquisire immagini in un singolo frame oppure in modo continuo. � poi possibile inserire il tempo di esposizione \textit{Exposure Time}, il numero di frame da catturare \textit{Frames to accumulate}, e modificare il valore \textit{Max. displayed value} del fondo scala. Un indicatore di immagine rende possibile la visualizzazione dell'immagine e un bottone booleano permette di salvare l'immagine. Infine troviamo un controllo per terminare il programma.

\subsubsection{Block Diagram}
\begin{figure} [h]
	\includegraphics[width=\linewidth]{img/SCOexample_bd.png}
	\caption{Block Diagram del progetto per sola fotocamera}
	\label{fig:midProj_camera_bd}
\end{figure}
Si pu� considerare il bock diagram, in figura \ref{fig:midProj_camera_bd}, idealmente diviso in tre parti che eseguono in senquenza: una parte iniziale di inizializzazione, una intermedia di funzionamento del programma, e una finale di chiusura graceful dell'applicazione.\\
Il primo subVI invocato � \emph{ConnectCamera.vi} e deve essere invocato prima che la fotocamera sia utilizzata, imposta la connessione della fotocamera dipendentemente dall'input immesso dall'utente; restituisce in uscita un cluster contenente tutte le informazioni che riguardano le impostazioni della fotocamera. Tale cluster di dati, insieme al cluster di errore, entra in una flat sequence che recupera le impostazioni memorizzate da un uso antecedente dello strumento e le mostra nel controlli appositi nel front panel; viene anche impostato il bottone booleano \textit{Live Mode} a false.\\
In seguito alla flat sequence troviamo un ciclo while con lo scopo di individuare il cambiamento dei bottoni e dei controlli. All'interno di questo ciclo sono presenti due case structure: la prima associata al bottone \textit{Save Image...} che permette di salvare l'immagine, la seconda associata all'\textit{or} fra il valore dei due bottoni \textit{Single Exposure} e \textit{Live Mode}. Viene eseguito quindi il codice all'interno di quest'ultima case structure qualora o un bottone o l'altro esclusivamente (grazie ad un'altra case structure interna) abbiamo il valore true. Una volta dentro la case structure viene eseguito un ciclo while all'interno del quale vengono invocate alcune API LabVIEW della fotocamera: dapprima \emph{SetExposureTime.vi} con la quale si setta il tempo di esposizione, poi \emph{GetExposure.vi} che restituisce la matrice di pixel corrispondente alla foto. Attraverso alcuni calcoli che tengono in considerazione anche il valore di fondo scala inserito viene generata l'immagine da visualizzare del controllo destinato. Il ciclo while interno si arresta se viene presa una singola immagine, oppure se non si � selezionata la modalit� Live Mode, o se si � premuto il bottone \textit{Exit} per uscire dal ciclo o quello di salvataggio dell'immagino, o ancora se si � verificato qualche errore. Il ciclo esterno si arresta invece se si � verificato qualche errore o se si � invocato il bottone \textit{Exit} per fermarlo.\\
All'esterno di tale ciclo viene eseguito l'ultimo frammento di codice: in una flat sequence vengono resettati i bottoni di \textit{Exit} e \textit{Live Mode}, impostandoli a false, e infine viene invocato l'ultimo subVI relativo alla fotocamera, \emph{ExitCamera.vi}, che ha il compito di rilasciare le risorse impegnate dal programma e arrestare la fotocamera nel modo corretto.

\section{Monocromatore}
Ad un primo tentativo di refactoring del progetto preesistente il codice relativo al controllo del monocromatore si � rivelato troppo caotico e sparso. Per questo si � realizzato un nuovo programma, prendendo ovviamente spunto dal precedente, dedicato solo al monocromatore, che rispettasse il corretto flusso dei dati e non facesse uso di flat sequence. Si descrivono di seguito front panel e block diagram di tale programma.

\subsection{Front Panel}
\begin{figure} [h]
	\includegraphics[width=0.5\linewidth]{img/ControlOnlyMono_Refactored.png}
	\caption{Front Panel del progetto per solo monocromatore}
	\label{fig:midProj_mono_fp}
\end{figure}
Gli elementi presenti nel front panel in figura \ref{fig:midProj_mono_fp} sono gli stessi dedicati al monocromatore nel precedente progetto, li rivedremo comunque brevemente. I controlli presenti sono:
\begin{itemize}
	\item \textit{Grating} per distinguere il tipo ti grating (1200 o 2400)
	\item \textit{Set Entrance Slit} per impostare la slit di entrata
	\item \textit{Set Wave Length} per inserire la lunghezza d'onda desiderata
\end{itemize}
Mentre gli indicatori sono:
\begin{itemize}
	\item \textit{Init} led che segnala se il monocromatore � stato inizializzato correttamente
	\item \textit{Error} led che segnala la presenza di eventuali errori in fase di inizializzazione
	\item \textit{Boot/Main Version} informazioni sulla versione del programma in uso
	\item \textit{Actual WL} lunghezza d'onda effettiva inserita
	\item \textit{Slit Width} slit di entrata inserita
	\item \textit{Settings Error} led che segnala un eventuale errore generico nel settaggio delle impostazioni del monocromatore
	\item \textit{Limits Hit} led che segnala l'inserimento di una lunghezza d'onda troppo alta
	\item \textit{Invalid Wavelength} led che segnala l'inserimento di una lunghezza d'onda non valida
\end{itemize}

\subsection{Block Diagram}
\begin{figure} [h]
	\includegraphics[width=\linewidth]{img/ControlOnlyMono_Refactored_bd.png}
	\caption{Block Diagram del progetto per solo monocromatore}
	\label{fig:midProj_mono_bd}
\end{figure}
Per la realizzazione del block diagram in figura \ref{fig:midProj_mono_bd} si � chiaramente preso spunto da quello gi� esistente, ma si � impiegato un consistente refactoring in particolare per eliminare le flat sequence.\\
Il primo subVI ad essere invocato � \emph{Start Up.vi} che porta i componenti interni del monocromatore nella posizione di default e restituisce il numero di versione del programma. Tale subVI restituisce anche un array di errori: esso viene ordinato in ordine crescente e ne viene estratto l'elemento in cima, se vi � un errore allora non � possibile continuare con il programma, in caso contrario si pu� procedere. Si entra quindi (in assenza di errori) nella case structure che permette di impostare come si desidera il monocromatore. I subVI che vengono invocati, in sequenza, sono: \emph{Port \& Grating.vi} per specificare il grating, \emph{Spectral GOTO.vi} per portare gli elementi interni alla lunghezza d'onda inserita dall'utente, \emph{Slits.vi} per inserire la slit di entrata e \emph{Spectral Position.vi} per completare gli indicatori corrispondenti. Chiaramente ognuno di questi subVI � inserito all'interno di una case structure e verrano eseguiti solo qualora il subVI precedente non abbia restituito in uscita errori. In questo modo si eliminano completamente le flat sequence e si sfrutta l'andamento del flusso dei dati.
\chapter{Realizzazione del progetto}
\thispagestyle{empty}

Gli strumenti coinvolti nel progetto sono quelli descritti nei capitoli precedenti, il monocromatore Jobin-Yvon HR460 e la fotocamera intensificata 4 Picos. Di seguito si descrive il programma realizzato per il controllo dei due strumenti attraverso un unico programma.

\section{Approccio iniziale}
Avendo davanti un progetto gi� scritto e funzionante, anche se per una fotocamera diversa, il primo approccio � stato quello di cercare di integrare il vecchio programma sostituendo i subVI della fotocamera precedente con quelli della fotocamera attuale. Tale tentativo si � rivelato fin dall'inizio fallimentare in quanto il codice del vecchio progetto risulta troppo caotico e avviluppato per essere modificato senza operare ingenti cambiamenti. Si � quindi optato per una riscrittura completa del codice prendendo sempre spunto da quello gi� scritto per le parti riutilizzabili.

\section{Struttura del progetto}
\textbf{Inserire foto del progetto .lvproj}
Per rendere il programma maggiormente organico e comprensibile si � deciso di creare un progetto LabVIEW: in questo modo i file e i subVI coinvolti sono raggruppati in un unico progetto, che risulta sicuramente pi� gestibile di un semplice insieme di file.\\
Si � creato all'inizio un VI principale, un \textit{main} che ha lo stesso scopo dei \textit{main} degli altri linguaggi di programmazione. In seguito si � cercato di creare pi� subVI possibile per rendere il programma leggibile e ordinato. Si descriveranno pi� avanti i subVI contenuti nel progetto e i loro scopi.

\section{Front Panel}
\textbf{Inserire foto del progetto .lvproj}
La struttura pensata per il front panel di questo progetto � quella dell'interfaccia grafica a \textit{tab} (controllo grafico di navigazione che permette all'utente di muoversi da un gruppo di controlli a un altro). Tale widget permette un'ideale suddivisione delle diverse aree di competenza del programma: di seguito verranno elencate con una descrizione degli elementi contenuti da ognuna.
\begin{itemize}
	\item \textit{Setting Mono} contiene i controlli necessari al settaggio del monocromatore. \textit{Grating} per impostare il grating (1200 o 2400) e rispettivi controlli per specificarne un eventuale calibrazione. \textit{Set Entrance Slit} e \textit{Set Wave Length} per impostare la slit di entrata e la lunghezza d'onda. \textit{Actual WL} e \textit{Slit Width} per visualizzare lunghezza d'onda e slit effettivamente inserite. Sono presenti inoltre tre led: \textit{Settings Error} indica un errore generico di settaggio, \textit{Invalid Wavelength} indica l'inserimento di una lunghezza d'onda non valida e \textit{Limits hit} evidenzia l'inserimento di un valore oltre i limiti consentiti.
	\item \textit{Acquire Data} contiene tutti i controlli per poter impostare la fotocamera e acquisire i dati. 
	\item \textit{Post Processing} permette di visualizzare lo spettro corrispondente all'immagine acquisita ed effettuare dei calcoli su tali dati. Con \textit{Waveform Integral} � possibile calcolare l'integrale della lunghezza d'onda su un certo intervallo. \textit{Peak Detector} consente di visualizzare il numero di picchi trovati al di sopra di una certa soglia di inserire e i loro valori. Attraverso il controllo \textit{Pixel To Average} � possibile scegliere il numero di pixel da con cui fare la media da visualizzare poi nel grafico. Sono stati mantenuti anche i controlli per acquisire il background e sottrarlo alla foto.
	\item \textit{Save} presenta due controlli booleani: \textit{Save} salva i dati be formattati in un file di testo che pu� essere aperto successivamente con altri editor, \textit{Save Image} salva invece l'immagine vera e propria.
\end{itemize}
Al di fuori di questo controllo � presente un pannello dedicato all'inizializzazione degli strumenti. Il led \textit{Init} segnala che il monocromatore � stato inizializzato senza errori e viene visulizzata anche la versione del programma, in caso contrario si attiva il led \textit{Error Occured}. Sono presenti anche due controlli per specificare la connessione che si intende utilizzare verso la fotocamera: \textit{USB} per connetterla al calcolatori attraverso cavo USB, \textit{CameraLink} o \textit{Analog} per altre modalit� di connessione non contemplate in questo progetto.\\
Infine il bottone \textit{Exit} consente di arrestare il programma nel modo corretto.

\section{Block Diagram}
\chapter{Conclusioni}
\thispagestyle{empty}

� stato realizzato il codice per il progetto richiesto: interfaccia grafica di controllo per una fotocamera intensificata e un monocromatore in ambiente LabVIEW. Per poter realizzare il programma � stata molto utile la partecipazione ad un corso su tale linguaggio per avere una conoscenza accademica oltre che concreta; tale corso ha anche permetto di conseguire la certificazione base di CLAD, Certified LabVIEW Associate Developer. 

\section{Risultati ottenuti}
Quasi tutte le specifiche richieste sono state soddisfatte. Il programma controlla correttamente entrambi gli strumenti e permette di salvare i dati acquisiti. Il monocromatore pu� essere settato correttamente controllando i dati prima di inviarli allo strumento, evitando cos� di danneggiare lo strumento. Per la fotocamera sono state riprese quasi tutte le opzioni presenti nel software proprietario. Nella fase di post processing dell'immagine � possibile visualizzare gli stessi calcoli presenti nel progetto preesistente.\\
La sottrazione del background deve ancora essere verificata in quanto la nuova fotocamera non d� la possibilit� di acquisire un background, � necessario effettuare questa operazione manualmente. Rispetto al progetto preesistente � stato eliminato il comando di \textit{Normalize Flat} in quanto considerato poco rilevante al fine degli scopi di ricerca.

\section{Sviluppi futuri}
Il programma pu� senza dubbio essere migliorato, sia dal punto di vista del front panel che del block diagram. Per quanto riguarda il front panel si tratta di una questione prettamente estetica, che collide anche con la versione di LabVIEW installata sul calcolatore dal momento che non in tutte le versione di LabVIEW sono presenti gli stessi controlli e indicatori. Il block diagram si pu� considerare organico e comprensibile, � comunque sempre possibile aggiungere nuove funzionalit� al programma, inclusa quella di \textit{Subtract Dark}. 

\appendix                               %imposta le appendici
\chapter{Prima Appendice}               %crea l'appendice
In questa Appendice non si \`e utilizzato il comando:\\
%%%%%%%%%%%%%%%%%%%%%%%%%%%%%%%%%%%%%%%%%\verb"" è equivalente all'
                                        %   ambiente verbatim,
                                        %   ma si utilizza all'interno
                                        %   di un discorso.
\verb"\clearpage{\pagestyle{empty}\cleardoublepage}", ed infatti
l'ultima pagina 8 ha l'intestazione con il numero di pagina in
alto.
%%%%%%%%%%%%%%%%%%%%%%%%%%%%%%%%%%%%%%%%%imposta l'intestazione di pagina
\rhead[\fancyplain{}{\bfseries \thechapter \:Prima Appendice}]
{\fancyplain{}{\bfseries\thepage}}
\chapter{Seconda Appendice}             %crea l'appendice
%%%%%%%%%%%%%%%%%%%%%%%%%%%%%%%%%%%%%%%%%imposta l'intestazione di pagina
\rhead[\fancyplain{}{\bfseries \thechapter \:Seconda Appendice}]
{\fancyplain{}{\bfseries\thepage}}
\begin{thebibliography}{90}             %crea l'ambiente bibliografia
\rhead[\fancyplain{}{\bfseries \leftmark}]{\fancyplain{}{\bfseries
\thepage}}
%%%%%%%%%%%%%%%%%%%%%%%%%%%%%%%%%%%%%%%%%aggiunge la voce Bibliografia
                                        %   nell'indice
\addcontentsline{toc}{chapter}{Bibliografia}
%%%%%%%%%%%%%%%%%%%%%%%%%%%%%%%%%%%%%%%%%provare anche questo comando:
%%%%%%%%%%%\addcontentsline{toc}{chapter}{\numberline{}{Bibliografia}}
\bibitem{K1} Primo oggetto bibliografia.
\bibitem{K2} Secondo oggetto bibliografia.
\bibitem{K3} Terzo oggetto bibliografia.
\bibitem{K4} Quarto oggetto bibliografia.
\end{thebibliography}
%%%%%%%%%%%%%%%%%%%%%%%%%%%%%%%%%%%%%%%%%non numera l'ultima pagina sinistra
\clearpage{\pagestyle{empty}\cleardoublepage}
\chapter*{Ringraziamenti}
\thispagestyle{empty}
Qui possiamo ringraziare il mondo intero!!!!!!!!!!\\
Ovviamente solo se uno vuole, non \`e obbligatorio.
\end{document}