\chapter{Analisi del progetto preesistente}
\thispagestyle{empty}
\lhead[\fancyplain{}{\bfseries\thepage}]{\fancyplain{}{\bfseries\rightmark}}
\pagenumbering{arabic} 

\section{Breve descrizione degli strumenti coinvolti}

\subsection{Monocromatore}
Un monocromatore è un dispositivo che scompone un singolo fascio di luce policromatica in più fasci di luce monocromatica (che contiene cioè onde di una sola frequenza), permettendo così di analizzare l'intensità in funzione della lunghezza d'onda.\\

Nello strumento la luce policromatica entra da una fessura; tramite un sistema ottico viene inviata su un reticolo di diffrazione o ad un prisma che scompone il fascio. Una seconda fenditura raccoglie poi il fascio di una determinata lunghezza d'onda.\\
In questo progetto si considera un monocromatore Jobin-Yvon HR460. La luce che entra dalla fessura viene rifratta (?) due volte attraverso due lenti tra cui è interposta una griglia che scompone il fascio di luce. 

\subsection{Fotocamera}
Una fotocamera intensificata è una fotocamera che al posto della pellicola fotosensibile utilizza un sensore (CCD) in grado di catturare l'immagine e trasformarla in un segnale elettrico di tipo analogico. Gli impulsi elettrici vengono convertiti in digitale da un convertitore analogico/digitale in un chip di elaborazione esterno al sensore. Viene quindi generato un flusso di dati digitali atti ad essere immagazzinati in vari formati su supporto di memoria.\\
Il CCD (Charged-Coupled Device) è un dispositivo attraverso il quale si ottiene un segnale elettrico in uscita, in seguito a una sequenza temporizzata di impulsi, grazie al quale è possibile ricostruire la matrice di pixel che compongono l'immagine proiettata sulla superficie del CCD stesso. Questa informazione può essere usata come segnale analogico, e quindi essere usata per riprodurre l'immagine su un monitor, oppure può essere convertita in formato digitale.
\section{Principali funzioni}


\subsection{Monocromatore}
\subsection{Fotocamera}

\section{Considerazioni sul progetto a livello di programmazione}

In pratica i difetti dai.