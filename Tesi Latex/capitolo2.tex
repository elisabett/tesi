\chapter{Analisi del progetto preesistente}

\lhead[\fancyplain{}{\bfseries\thepage}]{\fancyplain{}{\bfseries\rightmark}}
\pagenumbering{arabic} 

\section{Breve descrizione degli strumenti coinvolti}

\subsection{Monocromatore}
Un monocromatore è un dispositivo che scompone un singolo fascio di luce policromatica in più fasci di luce monocromatica (che contiene cioè onde di una sola frequenza), permettendo così di analizzare l'intensità in funzione della lunghezza d'onda.\\

Nello strumento la luce policromatica entra da una fessura; tramite un sistema ottico viene inviata su un reticolo di diffrazione o ad un prisma che scompone il fascio. Una seconda fenditura raccoglie poi il fascio di una determinata lunghezza d'onda.\\
In questo progetto si considera un monocromatore Jobin-Yvon HR460. 
\subsection{Fotocamera}

\section{Principali funzioni}

\subsection{Monocromatore}
\subsection{Fotocamera}

\section{Considerazioni sul progetto a livello di programmazione}

In pratica i difetti dai.