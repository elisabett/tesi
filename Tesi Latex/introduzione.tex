\chapter*{Introduzione}
\addcontentsline{toc}{chapter}{Introduzione}
Nell'ambito dello studio dei plasmi generati tramite scariche elettriche si rivela necessario catturare gli spettri generati da suddette interazioni per studiarne diverse propriet� quali composizione e temperatura. Per effettuare tali misurazioni sono necessari due strumenti: un monocromatore che scomponga e rifletta la fonte di luce in input e una fotocamera che catturi l'immagine dello spettro. \\
Risulta quindi necessaria un'interfaccia utente in grado di controllare simultaneamente i due strumenti in modo da facilitare il percorso di acquisizione e manipolazione degli spettri. Lo scopo della tesi � implementare l'interfaccia richiesta attraverso il linguaggio di programmazione grafica LabVIEW, con l'ausilio delle API dei due strumenti.\\
� presente un progetto preesistente che realizza l'interfaccia di controllo del monocromatore e di una fotocamera non intensificata. A partire da questo progetto se ne � realizzato uno nuovo che riutilizza alcune parti del precedente, considerando lo stesso monocromatore e una nuova fotocamera intensificata.\\
A partire quindi dal programma esistente si � effettuato uno studio dello stesso per poterlo cos� adattare alla nuova fotocamera. Dopo un primo tentativo di riutilizzo e refactoring si � preferito realizzare un programma ex novo, modificando anche alcuni elementi dell'interfaccia grafica. Chiaramente si sono mantenuti alcuni elementi, in particolare quelli legati al monocromatore, dal momento che lo strumento � rimasto lo stesso cos� come le librerie ad esso associate.