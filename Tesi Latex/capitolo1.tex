\chapter{Documento dei requisiti}
\label{capitolo1}
\thispagestyle{empty}

Si descrive nelle seguenti sezioni la specifica dei requisiti del sistema software che si intende sviluppare.

\section{Scopo}
Come accennato nell'introduzione si rende necessaria la realizzazione di una nuova interfaccia che possa controllare simultaneamente il monocromatore HR460 e la fotocamera intensificata 4 Picos. Il precedente programma aveva la capacit� di controllare lo stesso monocromatore ma una diversa fotocamera, SensiCam (version 3.0).

\section{Descrizione generale}
L'utente, attraverso il programma da progettare, deve poter controllare sia il monocromatore che la fotocamera intensificata. � importante che questi due strumenti possano essere controllati simultaneamente poich� le loro funzioni sono strettamente legate: il monocromatore genera un determinato spettro attraverso la sua struttura interna e la fotocamera deve poter catturare tale spettro il pi� precisamente possibile.

\subsection{Funzioni del prodotto}
Attraverso il programma deve essere possibile inizializzare le principali caratteristiche del monocromatore in base alle esigenze dell'utente: grating, lunghezza d'onda e fenditura di entrata. Deve anche essere possibile settare e modificare le impostazioni della fotocamera, compatibilmente con le funzioni messe a disposizione tramite le API della stessa. 

\subsection{Caratteristiche utente}
Il progetto da realizzare viene utilizzato da professori e dottorandi per la caratterizzazione spettroscopica di plasmi a bassa pressione e atmosferici. � necessario trovare un punto di equilibrio tra le esigenze dell'utente e i vincoli di programmazione.

\section{Requisiti specifici}

\subsection{Requisiti dell'interfaccia utente}
L'interfaccia deve risultare il pi� intuitiva possibile, per questo si deve cercare di mantenere un minimo di omogeneit� con il programma passato. Devono infatti coesistere nella stessa finestra sia i controlli per il monocromatore che quelli per la fotocamera.\\Vediamo quali sono i requisiti specifici per i due strumenti.

\begin{description}
  \item[Monocromatore] Per quanto riguarda la parte di interfaccia che permette di impostare correttamente il monocromatore � necessario poter modificare il grating (1200 o 2400) e la calibrazione di correzione della lunghezza d'onda. Si deve anche poter modificare la slit di entrata e la lunghezza d'onda \cite{monoManual};
  \item[Fotocamera] La parte di interfaccia che concerne la fotocamera intensificata deve contenere i seguenti parametri\cite{cameraManual} \cite{cameraAPIManual}:
	\begin{itemize}
\item MCP Gain Voltage: guadagno di tensione in volt tra $600 V$ e $950 V$
\item Exposure Time: tempo di esposizione, deve essere positivo e si deve poter specificare l'unit� di misura
\item Delay: ritardo in secondi dell'acquisizione dell'immagine
\item Frames to accumulate: numero di immagini da acquisire
\item Trigger Source: -Trig per far scattare l'otturatore esternamente attraverso l'input di trigger negativo collegato elettricamente ad un segnale, Fsync per far scattare l'otturatore internamente dal segnale Fsync
\item Gate Control: interno, l'input di controllo dell'otturatore � connesso internamente all'output IntGtP che pu� essere usato per scopi di controllo o trigger, o esterno, l'input di controllo dell'otturatore � collegato al connettore in input di ExtGtP
\item Start Option: Cold Start, non ci sono parametri pregressi da utilizzare, Warm Start, dopo il riavvio della fotocamera si chiede all'utente se si devono caricare i parametri usati precedentemente, Auto Warm Start, dopo il riavvio della fotocamera vengono caricati i parametri usati in precedenza senza la conferma da parte dell'utente
\item Detector: per specificare l'area attiva del chip CCD, frame intero, binning 2x2 oppure ROI (Region of Interest, solo 1/3 dell'area complessiva)
\item Digitalizer Mode: acquisizione dell'immagine con una precisione di 8 o 14 bit per pixel
\item CCD Gain: automatico oppure settabile manualmente (in $dB$)
\item Trigger Mode: trigger diretto (ad ogni segnale di trigger far� scattare l'otturatore) oppure singolo (in questo caso deve essere possibile inserire il numero di trigger per frame)
\item Modalit� di acquisizione dell'immagine: Single Exposure per catturare una sola immagine, Live Mode per acquisire immagini in modo continuo;
\end{itemize}
\end{description}
Devono anche essere presenti dei comandi per il post-processing e l'analisi delle immagini acquisite. Come nel progetto preesistente � necessario visualizzare lo spettro in un grafico che abbia nell'asse orizzontale il range di lunghezza d'onda osservato e nell'asse verticale l'intensit� relativa dello spettro. Inoltre deve essere possibile sottrarre un background acquisito in precedenza ed effettuare dei calcoli sullo spettro, come integrale sulla lunghezza d'onda, FWHM, picco massimo e numero di picchi. Alla fine delle operazioni, qualora ce ne sia la possibilit�, si deve poter salvare i dati in un file di testo che tenga traccia dei valori calcolati per visualizzare lo spettro.

\subsection{Linguaggio di programmazione}
\label{subsec:labview}
Il linguaggio di programmazione da utilizzare per realizzare il progetto richiesto � LabVIEW.\\ LabVIEW � un linguaggio di programmazione grafica (G - Graphical Programming Language) che utilizza un modello a flusso di dati invece di linee sequenziali di codice testuale, permettendo di scrivere codice funzionale utilizzando un layout grafico che assomiglia a un diagramma di flusso \cite{labviewWebSite}. \\
L'ambiente di programmazione LabVIEW presenta due principali finestre di lavoro: front panel e block diagram. 
\begin{description}
\item[Front Panel]Finestra in cui si visualizza e si modifica l'interfaccia utente. Sono disponibili diversi stili e comandi per l'interazione con l'utente. Si possono inserire bottoni, caselle di inserimento di stringhe o numeri, led, controlli a tab, indicatori per visualizzare grafici, immagini, matrici ecc.
\item[Block Diagram]Finestra in cui si compone il codice a blocchi con controlli, indicatori, subVI e molte alte componenti. Da tale finestra � possibile effettuare il debug del codice e controllare il contenuto delle variabili.
\end{description}
Le strutture principali utilizzate nel codice sono cicli, while e for, e case structure, che funzionano come if o switch. Per mantenere il valore delle variabili e condividerle con altre strutture si usano fili che le interconnettono. Elementi fondamentali nel linguaggio di programmazione grafica sono controlli e indicatori: i controlli permettono all'utente di inserire o modificare i dati, mentre gli indicatori consentono di visualizzare lo stato delle variabili.